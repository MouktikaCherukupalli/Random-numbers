\documentclass[journal,12pt,twocolumn]{IEEEtran}
\usepackage{setspace}
\usepackage{gensymb}
\usepackage{caption}
%\usepackage{multirow}
%\usepackage{multicolumn}
%\usepackage{subcaption}
%\doublespacing
\singlespacing
\usepackage{csvsimple}
\usepackage{amsmath}
\usepackage{multicol}
%\usepackage{enumerate}
\usepackage{amssymb}
%\usepackage{graphicx}
\usepackage{newfloat}
%\usepackage{syntax}
\usepackage{listings}

\usepackage{color}
\usepackage{tikz}
\usetikzlibrary{shapes,arrows}



%\usepackage{graphicx}
%\usepackage{amssymb}
%\usepackage{relsize}
%\usepackage[cmex10]{amsmath}
%\usepackage{mathtools}
%\usepackage{amsthm}
%\interdisplaylinepenalty=2500
%\savesymbol{iint}
%\usepackage{txfonts}
%\restoresymbol{TXF}{iint}
%\usepackage{wasysym}
\usepackage{amsthm}
\usepackage{mathrsfs}
\usepackage{txfonts}
\usepackage{stfloats}
\usepackage{cite}
\usepackage{cases}
\usepackage{mathtools}
\usepackage{caption}
\usepackage{enumerate}	
\usepackage{enumitem}
\usepackage{amsmath}
%\usepackage{xtab}
\usepackage{longtable}
\usepackage{multirow}
%\usepackage{algorithm}
%\usepackage{algpseudocode}
\usepackage{enumitem}
\usepackage{mathtools}
\usepackage{hyperref}
%\usepackage[framemethod=tikz]{mdframed}
\usepackage{listings}
    %\usepackage[latin1]{inputenc}                                 %%
    \usepackage{color}                                            %%
    \usepackage{array}                                            %%
    \usepackage{longtable}                                        %%
    \usepackage{calc}                                             %%
    \usepackage{multirow}                                         %%
    \usepackage{hhline}                                           %%
    \usepackage{ifthen}                                           %%
  %optionally (for landscape tables embedded in another document): %%
    \usepackage{lscape}     


\usepackage{url}
\def\UrlBreaks{\do\/\do-}


%\usepackage{stmaryrd}


%\usepackage{wasysym}
%\newcounter{MYtempeqncnt}
%\newcounter{MYtempeqncnt}
\DeclareMathOperator*{\Res}{Res}
%\renewcommand{\baselinestretch}{2}
\renewcommand\thesection{\arabic{section}}
\renewcommand\thesubsection{\thesection.\arabic{subsection}}
\renewcommand\thesubsubsection{\thesubsection.\arabic{subsubsection}}

\renewcommand\thesectiondis{\arabic{section}}
\renewcommand\thesubsectiondis{\thesectiondis.\arabic{subsection}}
\renewcommand\thesubsubsectiondis{\thesubsectiondis.\arabic{subsubsection}}

% correct bad hyphenation here
\hyphenation{op-tical net-works semi-conduc-tor}

%\lstset{
%language=C,
%frame=single, 
%breaklines=true
%}

%\lstset{
	%%basicstyle=\small\ttfamily\bfseries,
	%%numberstyle=\small\ttfamily,
	%language=Octave,
	%backgroundcolor=\color{white},
	%%frame=single,
	%%keywordstyle=\bfseries,
	%%breaklines=true,
	%%showstringspaces=false,
	%%xleftmargin=-10mm,
	%%aboveskip=-1mm,
	%%belowskip=0mm
%}
%\surroundwithmdframed[width=\columnwidth]{lstlisting}
\def\inputGnumericTable{}                                 %%
\lstset{
%language=C,
frame=single, 
breaklines=true,
columns=fullflexible
}
 

\begin{document}
%
\tikzstyle{block} = [rectangle, draw,
    text width=3em, text centered, minimum height=3em]
\tikzstyle{sum} = [draw, circle, node distance=3cm]
\tikzstyle{input} = [coordinate]
\tikzstyle{output} = [coordinate]
\tikzstyle{pinstyle} = [pin edge={to-,thin,black}]
\theoremstyle{definition}
\newtheorem{theorem}{Theorem}[section]
\newtheorem{problem}{Problem}
\newtheorem{proposition}{Proposition}[section]
\newtheorem{lemma}{Lemma}[section]
\newtheorem{corollary}[theorem]{Corollary}
\newtheorem{example}{Example}[section]
\newtheorem{definition}{Definition}[section]
%\newtheorem{algorithm}{Algorithm}[section]
%\newtheorem{cor}{Corollary}
\newcommand{\BEQA}{\begin{eqnarray}}
\newcommand{\EEQA}{\end{eqnarray}}
\newcommand{\define}{\stackrel{\triangle}{=}}
\bibliographystyle{IEEEtran}
%\bibliographystyle{ieeetr}
\providecommand{\nCr}[2]{\,^{#1}C_{#2}} % nCr
\providecommand{\nPr}[2]{\,^{#1}P_{#2}} % nPr
\providecommand{\mbf}{\mathbf}
\providecommand{\pr}[1]{\ensuremath{\Pr\left(#1\right)}}
\providecommand{\qfunc}[1]{\ensuremath{Q\left(#1\right)}}
\providecommand{\sbrak}[1]{\ensuremath{{}\left[#1\right]}}
\providecommand{\lsbrak}[1]{\ensuremath{{}\left[#1\right.}}
\providecommand{\rsbrak}[1]{\ensuremath{{}\left.#1\right]}}
\providecommand{\brak}[1]{\ensuremath{\left(#1\right)}}
\providecommand{\lbrak}[1]{\ensuremath{\left(#1\right.}}
\providecommand{\rbrak}[1]{\ensuremath{\left.#1\right)}}
\providecommand{\cbrak}[1]{\ensuremath{\left\{#1\right\}}}
\providecommand{\lcbrak}[1]{\ensuremath{\left\{#1\right.}}
\providecommand{\rcbrak}[1]{\ensuremath{\left.#1\right\}}}
\theoremstyle{remark}
\newtheorem{rem}{Remark}
\newcommand{\sgn}{\mathop{\mathrm{sgn}}}
\providecommand{\abs}[1]{\left\vert#1\right\vert}
\providecommand{\res}[1]{\Res\displaylimits_{#1}} 
\providecommand{\norm}[1]{\left\Vert#1\right\Vert}
\providecommand{\mtx}[1]{\mathbf{#1}}
\providecommand{\mean}[1]{E\left[ #1 \right]}
\providecommand{\fourier}{\overset{\mathcal{F}}{ \rightleftharpoons}}
%\providecommand{\hilbert}{\overset{\mathcal{H}}{ \rightleftharpoons}}
\providecommand{\system}{\overset{\mathcal{H}}{ \longleftrightarrow}}
	%\newcommand{\solution}[2]{\textbf{Solution:}{#1}}
\newcommand{\solution}{\noindent \textbf{Solution: }}
\newcommand{\myvec}[1]{\ensuremath{\begin{pmatrix}#1\end{pmatrix}}}
\providecommand{\dec}[2]{\ensuremath{\overset{#1}{\underset{#2}{\gtrless}}}}
\DeclarePairedDelimiter{\ceil}{\lceil}{\rceil}
%\numberwithin{equation}{section}
%\numberwithin{problem}{subsection}
%\numberwithin{definition}{subsection}
\makeatletter
\@addtoreset{figure}{section}
\makeatother
\let\StandardTheFigure\thefigure
%\renewcommand{\thefigure}{\theproblem.\arabic{figure}}
\renewcommand{\thefigure}{\thesection}
%\numberwithin{figure}{subsection}
%\numberwithin{equation}{subsection}
%\numberwithin{equation}{section}
%\numberwithin{equation}{problem}
%\numberwithin{problem}{subsection}
\numberwithin{problem}{section}
%%\numberwithin{definition}{subsection}
%\makeatletter
%\@addtoreset{figure}{problem}
%\makeatother
\makeatletter
\@addtoreset{table}{section}
\makeatother
\let\StandardTheFigure\thefigure
\let\StandardTheTable\thetable
\let\vec\mathbf
\numberwithin{equation}{section}
\vspace{3cm}
\title{%Convex Optimization in Python
	\logo{
	Random Numbers
	}
}
%\title{
%	\logo{Matrix Analysis through Octave}{\begin{center}\includegraphics[scale=.24]{tlc}\end{center}}{}{HAMDSP}
%}
% paper title
% can use linebreaks \\ within to get better formatting as desired
%\title{Matrix Analysis through Octave}
%
%
% author names and IEEE memberships
% note positions of commas and nonbreaking spaces ( ~ ) LaTeX will not break
% a structure at a ~ so this keeps an author's name from being broken across
% two lines.
% use \thanks{} to gain access to the first footnote area
% a separate \thanks must be used for each paragraph as LaTeX2e's \thanks
% was not built to handle multiple paragraphs
%
\author{ AI21BTECH11007}$
% note the % following the last \IEEEmembership and also \thanks - 
% these prevent an unwanted space from occurring between the last author name
% and the end of the author line. i.e., if you had this:
% 
% \author{....lastname \thanks{...} \thanks{...} }
%                     ^------------^------------^----Do not want these spaces!
%
% a space would be appended to the last name and could cause every name on that
% line to be shifted left slightly. This is one of those "LaTeX things". For
% instance, "\textbf{A} \textbf{B}" will typeset as "A B" not "AB". To get
% "AB" then you have to do: "\textbf{A}\textbf{B}"
% \thanks is no different in this regard, so shield the last } of each \thanks
% that ends a line with a % and do not let a space in before the next \thanks.
% Spaces after \IEEEmembership other than the last one are OK (and needed) as
% you are supposed to have spaces between the names. For what it is worth,
% this is a minor point as most people would not even notice if the said evil
% space somehow managed to creep in.
% The paper headers
%\markboth{Journal of \LaTeX\ Class Files,~Vol.~6, No.~1, January~2007}%
%{Shell \MakeLowercase{\textit{et al.}}: Bare Demo of IEEEtran.cls for Journals}
% The only time the second header will appear is for the odd numbered pages
% after the title page when using the twoside option.
% 
% *** Note that you probably will NOT want to include the author's ***
% *** name in the headers of peer review papers.                   ***
% You can use \ifCLASSOPTIONpeerreview for conditional compilation here if
% you desire.
% If you want to put a publisher's ID mark on the page you can do it like
% this:
%\IEEEpubid{0000--0000/00\$00.00~\copyright~2007 IEEE}
% Remember, if you use this you must call \IEEEpubidadjcol in the second
% column for its text to clear the IEEEpubid mark.
% make the title area
\maketitle
\tableofcontents
\bigskip
\renewcommand{\thefigure}{\theenumi}
\renewcommand{\thetable}{\theenumi}
\begin{abstract}
This manual provides a simple introduction to the generation of random numbers
\end{abstract}
%%
\section{Uniform Random Numbers}
Let $U$ be a uniform random variable between 0 and 1.
\begin{enumerate}[label=\thesection.\arabic*
,ref=\thesection.\theenumi]
\item Generate $10^6$ samples of $U$ using a C program and save into a file called uni.dat .
\\
\solution Download the following files and execute the  C program.
\begin{lstlisting}
wget https://github.com/MouktikaCherukupalli/Random_Numbers/blob/main/codes/1.1.c
wget https://github.com/MouktikaCherukupalli/Random_Numbers/blob/main/codes/coeffs.h
\end{lstlisting}
%
\item
Load the uni.dat file into python and plot the empirical CDF of $U$ using the samples in uni.dat. The CDF is defined as
\begin{align}
F_{U}(x) = \pr{U \le x}
\end{align}
\\
\solution  The following code plots Fig. \ref{fig:uni_cdf}
\begin{lstlisting}
wget https://github.com/MouktikaCherukupalli/Random_Numbers/blob/main/codes/1.2.py
\end{lstlisting}
\begin{figure}
\centering
\includegraphics[width=\columnwidth]{uni_cdf.pdf}
\caption{The CDF of $U$}
\label{fig:uni_cdf}
\end{figure}
%
\item
Find a  theoretical expression for $F_{U}(x)$.
\solution The PDF of $U$ is given by
	\begin{align}
		p_{U}(x) = 
		\begin{cases}
			1 & x \in [0, 1] \\
			0 & \text{otherwise}
		\end{cases}
	\end{align}
The CDF of $U$ is given by
	\begin{align}
		F_{U}(x) = \pr{U \le x} = \int_{-\infty}^x p_{U}(x) ~\mathrm{d}x
	\end{align}
	
	If $x<0$,
	\begin{align}
		\int_{-\infty}^x p_{U}(x) ~\mathrm{d}x = \int_{-\infty}^x 0 ~\mathrm{d}x = 0
	\end{align}
	
	If $0<x<1$,
	\begin{align}
		\int_{-\infty}^x p_{U}(x) ~\mathrm{d}x &= \int_{-\infty}^0 0 ~\mathrm{d}x + \int_0^x 1 ~\mathrm{d}x \\
		&= 0 + x \\
		&= x
	\end{align}
	
	If $x>1$,
	\begin{multline}
		\int_{-\infty}^x p_{U}(x) ~\mathrm{d}x \\= \int_{-\infty}^0 0 ~\mathrm{d}x + \int_0^1 1 ~\mathrm{d}x +  \int_1^x 0 ~\mathrm{d}x 
	\end{multline}
	\begin{align}
		\int_{-\infty}^x p_{U}(x) ~\mathrm{d}x &= 0 + 1 + 0 \\
		&= 1
	\end{align}
	
	Therefore, we obtain the CDF of $U$ as
	\begin{align}
		F_{U}(x) = 
		\begin{cases}
			0 & x < 0 \\
			x & 0 \le x \le 1 \\
			1 & x > 1
		\end{cases}
	\end{align}
\item The mean of $U$ is defined as
	\begin{align}
	\mean{U} = \frac{1}{N}\sum_{i=1}^{N}U_i
	\end{align}
	and its variance as
\begin{align}
		\text{Var}\sbrak{U} = \mean{U- \mean{U}}^2 
\end{align}
	Write a C program to  find the mean and variance of $U$
	
	\solution Download the C source code by executing the following commands
	\begin{lstlisting}
wget https://github.com/MouktikaCherukupalli/Random_Numbers/blob/main/codes/1.4.c
	\end{lstlisting}
	Compile and run the C program by executing the following
	\begin{lstlisting}
		gcc 1.4.c -lm
		./a.out
	\end{lstlisting}
	The output of the code is
	\begin{align}
		\mu_{\text{emp}} &= 0.500007 \\
		\sigma_{\text{emp}}^2 &= 0.083301 
	\end{align}
\item Verify your result theoretically given that
\end{enumerate}
%
\begin{equation}
E\sbrak{U^k} = \int_{-\infty}^{\infty}x^kdF_{U}(x)dx
\end{equation}
\solution Verifying result theoritically\\
Given that
\begin{align}
    E[U^k] &= \int_{-\infty}^{\infty} x^k dF_U(x)
\end{align}
Mean is given by
\begin{align}
    E[U] &= \int_{-\infty}^{\infty} x dF_U(x) \\
    &= \int_{-\infty}^{\infty} x dx \\
    &= \bigg[\frac{x^2}{2}\bigg]_0^1 \\
    &= \frac{1}{2}
\end{align}
Varaiance is given by
\begin{align}
    E[U-E[U]]^2 &= E[U^2] - [E[U]]^2 \\
    E[U]^2 &= \int_{-\infty}^{\infty} x^2 dF_U(x)\\
    &= \int_{-\infty}^{\infty} x^2 dx \\
    &= \bigg[\frac{x^3}{3}\bigg]_0^1 \\
    &= \frac{1}{3} \\
    E[U^2] - [E[U]]^2&= \frac{1}{3} - (\frac{1}{2})^2 \\
    &= \frac{1}{3} - \frac{1}{4} \\
    &= \frac{1}{12}
\end{align}
\end{enumerate}
\section{Central Limit Theorem}

	\begin{enumerate}[label=\thesection.\arabic*,ref=\thesection.\theenumi]
	\item Generate $10^6$ samples of the random variable
	\begin{align}
		X = \sum_{i=1}^{12}U_i -6
	\end{align}

	using a C program, where $U_i, i = 1,2,\dots, 12$ are  a set of independent uniform random variables between 0 and 1 and save in a file called gau.dat
	
	\solution Download the C source code by executing the following commands
	\begin{lstlisting}
		wget https://github.com/MouktikaCherukupalli/Random_Numbers/blob/main/codes/2.1.c
		wget https://github.com/MouktikaCherukupalli/Random_Numbers/blob/main/codes/coeffs.h
	\end{lstlisting}
	Compile and run the C program by executing the following
	\begin{lstlisting}
		gcc 2.1.c -lm
		./a.out
	\end{lstlisting}
\item Load gau.dat in Python and plot the empirical CDF of $X$ using the samples in gau.dat. What properties does a CDF have?

	\solution Download the following Python code that plots Fig. \ref{fig-2.2}
	\begin{lstlisting}
		wget https://github.com/MouktikaCherukupalli/Random_Numbers/blob/main/codes/2.2.py
	\end{lstlisting}
	Run the code by executing
	\begin{lstlisting}
		python 2.2.py
	\end{lstlisting}
	\begin{figure}
		\centering
		\includegraphics[width=\columnwidth]{gau_cdf.pdf}
		\caption{The CDF of $X$}
		\label{fig-2.2}
	\end{figure}
Every cdf is non decreasing function and bounded between 0 and 1
\item Load gau.dat in Python and plot the empirical PDF of $X$ using the samples in gau.dat. The PDF of $X$ is defined as
	\begin{align}
	p_{X}(x) &= \frac{d}{dx}F_{X}(x)
	\end{align}
	What properties does the PDF have?
	\solution Download the following Python code that plots Fig. \ref{fig-2.2}
	\begin{lstlisting}
		wget https://github.com/MouktikaCherukupalli/Random_Numbers/blob/main/codes/2.3.py
	\end{lstlisting}
		\begin{figure}
		\centering
		\includegraphics[width=\columnwidth]{gau_pdf.pdf}
		\caption{The PDF of $X$}
		\label{fig-2.3}
	\end{figure}
pdf is bounded between 0 and 1 
\item Find the mean and variance of $X$ by writing a C program
	
	\solution Download the C source code by executing the following commands
	\begin{lstlisting}
	wget https://github.com/MouktikaCherukupalli/Random_Numbers/blob/main/codes/2.4.c
	\end{lstlisting}
	Compile and run the C program by executing the following
	\begin{lstlisting}
		gcc 2.4.c -lm
		./a.out
	\end{lstlisting}
	The output of the code is
	\begin{align}
		\mu_{\text{emp}} &= 0.000294 \\
	\sigma_{\text{emp}}^2 &= 0.999560 
	\end{align}	

	\item Given that 
	\begin{align}
		p_{X}(x) = \frac{1}{\sqrt{2\pi}}\exp\brak{-\frac{x^2}{2}}, -\infty < x < \infty,
	\end{align}
	repeat the above exercise theoretically
	
	\solution The mean of $X$ is given by
	\begin{align}
		\mean{X} &= \int_{-\infty}^{\infty} x p_{X}(x) \mathrm{d}x \\
		&= \int_{-\infty}^{\infty} \frac{x}{\sqrt{2\pi}}\exp\brak{-\frac{x^2}{2}} \mathrm{d}x 
	\end{align}
	if we consider 
	\begin{align} 
		  \dfrac{x}{\sqrt{2\pi}}\exp\brak{-\frac{x^2}{2}} &=f(x)\\
		\implies f(-x) &= \dfrac{-x}{\sqrt{2\pi}}\exp\brak{-\frac{(-x)^2}{2}} \\
		&= - \dfrac{x}{\sqrt{2\pi}}\exp\brak{-\frac{x^2}{2}} \\
		&= - f(x)
	\end{align}
$\therefore$ $f(x)$ is an odd function\\
we know that for an odd function,
	\begin{align}
	\int_{-\infty}^{\infty} f(x) \mathrm{d}x &= 0
		\implies 
		\mean{X} &=0
	\end{align}
for variance
	\begin{align}
		\mean{X^2} &= \int_{-\infty}^{\infty} x^2 p_{X}(x) \mathrm{d}x \\
		&= \int_{-\infty}^{\infty} \frac{x^2}{\sqrt{2\pi}}\exp\brak{-\frac{x^2}{2}} \mathrm{d}x 
\end{align}
integration by parts,
	\begin{align}
		\mean{X^2} = \sqrt{\frac{1}{2\pi}}  \int_{-\infty}^{\infty} x \cdot x \exp\brak{-\frac{x^2}{2}} \mathrm{d}x 
	\end{align}
\begin{multline}
		= \sqrt{\frac{1}{2\pi}} \brak{\left. x \int x \exp\brak{-\frac{x^2}{2}} \mathrm{d}x}\right|_{-\infty}^{\infty} \\- \sqrt{\frac{1}{2\pi}}  \int_{-\infty}^{\infty} 1 \cdot \int x \exp\brak{-\frac{x^2}{2}} \mathrm{d}x
\end{multline}
Substituting $t = -\frac{x^2}{2} \implies \mathrm{d}t = -x\mathrm{d}x$
	\begin{align}
		\int x \exp\brak{-\frac{x^2}{2}} \mathrm{d}x &= \int -\exp(t) \mathrm{d}{t} \\
		&= - \exp(t) \\
		&= - \exp\brak{-\frac{x^2}{2}}
	\end{align}
\begin{align}
		\left. -x \exp\brak{-\frac{x^2}{2}} \right|_{-\infty}^{\infty} = 0 - 0 = 0 \\
		as, \lim_{x\to\infty} x \exp\brak{-\frac{x^2}{2}}  =0\\
		and \lim_{x\to-\infty} x \exp\brak{-\frac{x^2}{2}}  =0
	\end{align}
Also, 
	\begin{align}
		&\int_{-\infty}^{\infty} - \exp\brak{-\frac{x^2}{2}} \mathrm{d}x\\
\text{by substituting} \frac{x^2}{2} &= t^2 \\
		&= -{\sqrt{2}} \int_{-\infty}^{\infty} \exp(-t^2) \mathrm{d}t \\
		&= -{\sqrt{2}} \sqrt{\pi} \\
		&= - \sqrt{2\pi}
	\end{align}
now,
	\begin{align}
		\mean{X^2} &= 0 - \sqrt{\frac{1}{2\pi}} \brak{- \sqrt{2\pi}} \\
		&= 1 \\
		\therefore \text{var}\sbrak{X} &= \mean{X^2} - \brak{\mean{X}}^2 \\
		&= 1 - 0 \\
		&= 1
	\end{align}
\end{enumerate}
\section{From Uniform to Other}
	\begin{enumerate}[label=\thesection.\arabic*,ref=\thesection.\theenumi]
	\item Generate samples of 
	\begin{align}
		V = -2\ln\brak{1-U}
	\end{align}
	and plot its CDF
	
	\solution Download the C source code by executing the following commands
	\begin{lstlisting}
		wget https://github.com/MouktikaCherukupalli/Random_Numbers/blob/main/codes/3.1.c
	\end{lstlisting}
	Compile and run the C program by executing the following
	\begin{lstlisting}
		gcc 3.1.c -lm
		./a.out
	\end{lstlisting}
	Download the following Python code that plots Fig. \ref{fig-3.1}
	\begin{lstlisting}
		wget https://github.com/MouktikaCherukupalli/Random_Numbers/blob/main/codes/3.1.py
	\end{lstlisting}
	\begin{figure}
		\centering
		\includegraphics[width=\columnwidth]{var_cdf.pdf}
		\caption{The CDF of $V$}
		\label{fig-3.1}
	\end{figure}
		
	\solution We have
	\begin{align}
		F_V(x) &= \pr{V \le x} \\
		&= \pr{-2\ln\brak{1-U} \le x} \\
		&= \pr{\ln\brak{1-U} \ge -\frac{x}{2}} \\
		&= \pr{1-U \ge \exp\brak{-\frac{x}{2}}} \\
		&= \pr{U \le 1 - \exp\brak{-\frac{x}{2}}} \\
		&= F_U\brak{1 - \exp\brak{-\frac{x}{2}}}
	\end{align}
	we know that ,
	\begin{align}
		F_{U}(x) = 
		\begin{cases}
			0 & x < 0 \\
			x & 0 \le x \le 1 \\
			1 & x > 1
		\end{cases}
	\end{align}
now,
	\begin{align}
		0 \le 1-\exp\brak{-\frac{x}{2}} &< 1 \qquad \text{if } x \ge 0	\\	
		1-\exp\brak{-\frac{x}{2}} &< 0 \qquad \text{if } x < 0	
	\end{align}
	
	Therefore,
	\begin{align}
		F_V(x) = 
		\begin{cases}
			1-\exp\brak{-\dfrac{x}{2}} & x \ge 0 \\
			0 & x < 0
		\end{cases}
	\end{align}
	\end{enumerate}
\end{enumerate}
\end{document}
